%Titlepage

\thispagestyle{empty}
\hbox{\ }
\vspace{1in}
\renewcommand{\baselinestretch}{1}
\small\normalsize
\begin{center}

\large{{\textbf{Manual : HYDRA}}}\\
\textbf{Rotorcraft Conceptual Sizing}\\
\ \\
\large{ }
\ \\
\textbf{Alfred Gessow Rotorcraft Center} \\
\textbf{Department of Aerospace Engineering} \\
\textbf{University of Maryland, College Park}\\
\ \\
\ \\
\textbf{Authors}
\ \\
\ \\
Bharath Govindarajan \\
Ananth Sridharan \\
\textbf{} \\
\end{center}
\noindent
This is the theory and user manual for the rotorcraft sizing code \textbf{HYDRA}, based on a revamped expansion of Bharath Govindarajan's conceptual sizing code developed for the 2013 AHS  Student Design Competition at the University of Maryland. This manual contains a description of the theory and various operations performed by the sizing code for both conventional and novel rotorcraft.
\\
\\
\large{\textbf{Design Philosophies }} \\
\normalsize
\\
HYDRA is written entirely in Python so that the baseline analysis can be easily modified for custom applications where required. The compute-heavy modules are implemented in Fortran to increase the speed of the calculations. Further, HYDRA can call the comprehensive analysis PRASADUM (with its built in blade deflection and integrated Maryland Free Wake) to calculate rotor power in all flight conditions. yHYDRA was originally designed to find the best design with full parameter space exploration, i.e., "carpet bombing". Python is the baseline language for this code, because HYDRA is intended to be used as a flexible analysis tool. The more time you, as a developer, spend trying to understand, fix and modify code, the less time you have to actually do something with it.

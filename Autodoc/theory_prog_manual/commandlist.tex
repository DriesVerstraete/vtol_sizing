\newenvironment{Figure}
  {\par\medskip\noindent\minipage{\linewidth}}
  {\endminipage\par\medskip}
\setlength{\textwidth}{5.9in}
\setlength{\textheight}{9in}
\setlength{\topmargin}{-.50in}
%\setlength{\topmargin}{0in}    %use this setting if the printer makes the the top margin 1/2 inch instead of 1 inch
\setlength{\oddsidemargin}{.55in}
\setlength{\parindent}{.4in}

%get a bold face "R" with arguments)
\newcommand{\arr}[3]{\textbf{R}\left(#1,#2,#3\right)}

%small bold "r" with vector
\newcommand{\aar}{\textbf{r}}

%double underscore with \textrm
\newcommand{\ud}[1]{_{_\textrm{#1}}}

%single space with \textrm{}
\newcommand{\spc}{\textrm{ }}

%\newcommand\pmat[4]{\ensuremath{\begin{pmatrix} #1 & #2 \\ #3 & #4\end{pmatrix}}}

%get a bold face "T"
\newcommand{\tee}{\textbf{T}}

%^T symbol for transpose
\newcommand{\tr}{^{^\textrm{T}}}

%short notation for denoting a vector			    % removed overline
\renewcommand{\vector}[1]{\textbf{#1}}

%short notation for a vector with greek stuff		% removed overline
\newcommand{\grkvec}[1]{\boldsymbol{#1}}

%To obtain a character with bold and italics
\newcommand{\qvec}[1]{\textbf{\textit{#1}}}

% Place a "hat" with italics around a bold italicized letter (see prev for \qvec)
\newcommand{\ihat}[1]{\mathit{{\qvec{#1}}}}			% removed hat after mathit
\newcommand{\ihatpr}[1]{\mathit{{\qvec{#1}^\prime}}} 		% removed hat

% Write a unit vector "i,j,k" with the required suffix
\newcommand{\pmat}[1]{\ensuremath{\begin{Bmatrix} \ihat{i}_{_\textrm{#1}} \\ \ihat{j}_{_\textrm{#1}} \\ \ihat{k}_{_\textrm{#1}} \end{Bmatrix}}}

% Write a unit vector "i,j,k" with the required suffix in a row vector
\newcommand{\hmat}[1]{\ensuremath{\begin{Bmatrix} \ihat{i}_{_\textrm{#1}} & \ihat{j}_{_\textrm{#1}} & \ihat{k}_{_\textrm{#1}} \end{Bmatrix}}}

% Write a unit vector "i,j,k" 
\newcommand{\qmat}[1]{\ensuremath{\begin{Bmatrix} \ihat{i}^{#1} \\ \ihat{j}^{#1} \\ \ihat{k}^{#1} \end{Bmatrix}}}

% Write a unit vector "i,j,k" with suffixes
\newcommand{\rmat}[2]{\ensuremath{\begin{Bmatrix} \ihat{i}^{#1}{#2} \\ \ihat{j}^{#1}{#2} \\ \ihat{k}^{#1}{#2} \end{Bmatrix}}}

%to represent undeformed frame vector quantities
\newcommand{\udvec}[1]{\widetilde{\textbf{#1}}}

%HYDRA
\newcommand{\ty}[1]{\texttt{#1}}

\newcommand{\blue}[1]{\textcolor{blue}{\textbf{#1}}}

\newcommand{\red}[1]{\textcolor{red}{\textbf{#1}}}

\newcommand{\hydra}{\ty{HYDRA}}

\newcommand{\python}{\ty{Python}}

\newcommand{\pls}{\spc+\spc}

%====================================================
\newcommand{\tbsp}{\rule{0pt}{18pt}} %used to get a vertical distance after \hline
\renewcommand{\baselinestretch}{2}

\lstset{escapeinside={<@}{@>}}
